\documentclass[12pt]{article}

\usepackage{amsmath}
\usepackage[margin=1.0in]{geometry}

\usepackage{Sweave}
\begin{document}
\Sconcordance{concordance:data_used_us.tex:data_used_us.Rnw:%
1 5 1 1 0 24 1}


\section{Data Used (US)}

For all data the earliest reference data used was 198703 and the latest
reference date used was 201503. 

\subsection{Yield rates}
The US federal reserve provides yield rates on bond data. The one and ten year
bond yields were used. These are provided as $i^{(2)}$ from which an effective
annual rate $i$ is calculated and a force of interest $\delta$ is calculated.
The quarterly force of interest is $\delta/4$.

\subsection{Stock index}
The SP500 was used from Yahoo finance (index $\wedge$GRSP). Suppose we know the stock price
at 198610, 198611, 198612, 198701, 198702, 198703, 198704 (start of month) and we want the return over the quarter 1 of 1987. Intuitively, If you bought a stock at 198701 (start of month) and sold it at 198704 (start of month) then you would have made a profit (negative or positive) of $S_{1987-04} - S_{1987-01}$ and your return over quarter 1 would be $\alpha = S_{1987-04}/S_{1987-01} - 1$ implying a continuously compounded return over the quarter of $\delta = \ln(1+\alpha)$.

\subsection{Inflation}
U.S. inflation information was obtained from the misery index at miseryindex.us.

\subsection{Salary}
Salary information was obtained from the U.S. department of commerce (Bureau of Economic Analysis) at bea.gov - per capita GDP was used.

\end{document}
