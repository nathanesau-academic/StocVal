\documentclass[12pt]{article}

\usepackage[margin=1.0in]{geometry}
\usepackage{amsmath}

\usepackage{Sweave}
\begin{document}
\Sconcordance{concordance:data_used.tex:data_used.Rnw:%
1 5 1 1 0 72 1}


\section{Data used (Canada)}

\subsection{Yield rates}
Statistics Canada's socio-economic database CANSIM provides Bank of Canada
Benchmark Bond Yields and Canadian treasury bill yields. These were given on a
daily basis - the earliest reference data avaiable was 199501 and the latest
reference data available was 201506. To be consistent with the inflation
reference dates, only data until 201505 was used. The vectors used were \textbf{V39051,
V39052, V39053, V39054, V39055, V39063, V39067}.

\medskip
The following steps were taken to obtain monthly data. 
\begin{enumerate}
	\item All rows where the yield was \texttt{NA} or \texttt{0.00} were
		removed.
	\item The average yield over a month was calculated using the daily data
		available (note that information for every day was not available in a
		given month).
\end{enumerate}

For treasury bills (1 month, 1 year), let $y$ represent the yield rate. The
yield rate is quoted under simple interest. Therefore, the return on the
treasury bill is $yt$ where $t$ is the length of the treasury bill, and the
price is $1 / (1+yt)$.

\medskip
For government of Canada bonds (2 year, 3 year, 5 year, 7 year, 10 year), the yield rate is compounded semi-annually. Let
$i^{(2)}$ represent the yield rate. The effective annual rate is $i = (1+
i^{(2)}/2)^2 - 1$ and the
price is $v^{n}$ where $v = 1/(1+i)^n$

\medskip
Each yield rate was converted to a continuously compounded rate. For the treasury
bills, let $i$ represent the effective annual yield rate under compound
interest. Then $i = (1+yt)^{(1/t)} - 1$ where $y$ is the yield rate under simple
interest. The force of interest is $\delta = \ln(1+i)$. For the government of
Canada bonds, the force of interest is $\delta = \ln(1+i)$. We want the
continuous rate over a monthly interval not over a yearly interval, so each of
the force of interest terms were divided by 12.

\subsection{Stock index}
From yahoo finance, we can get the monthly stock prices for the index
\texttt{S\&P/TSX} which accounts for 95\% of the Canadian equities market. To
calculate the stock return over a monthly interval we take $\alpha_{1/12} = S_{1/12} / S_{0}$
then $\alpha_{2/12} = S_{2/12} / S_{1/12}$ and so on. Then the force of interest
over a monthly interval is $\delta = \ln(1+\alpha_i)$. 

\medskip
The earliest reference date used was 199501 so that the dates lined up with the
yield rates. Dividends were not modeled, only the stock index. However,
reference dates are available into the 1970s.

\subsection{Inflation}
The Canadian consumer price index is also available from CANSIM. The vector used
was \textbf{V41755375}, which provides the CPI from 198401 onwards until
201505. To be consistent with the yield rate reference dates only data from
199501 onwards was used.

The inflation rate over a month was calculated by dividing the CPI over the CPI
of the previous month and subtracting 1. The continuously compounded inflation
rate over the month was calculated by taking $\ln(1+\pi)$ where $\pi$ is the
inflation rate over the month.

\subsection{Salary}
The seasonally adjusted annual salary rates are also available from CANSIM. The
	vector used was \textbf{V62468795}. However, the data frequency available is
	quarterly, so currently salary is not being included in the model. Monthly data
    is only available from 2001 onwards from the survey of employment, payrolls and hours (SEPH).

\end{document}
